\documentclass{article}
\usepackage{quad}
\usepackage[english,russian]{babel}
\usepackage[utf8]{inputenc}

\begin{document}

\title{В поисках пятиточечного условия\\ Александровского типа}
\author{Антон Петрунин}
\date{}
\maketitle
\begin{abstract}
Приведено описание четырёх-точеного условия Александрова через квадратичную форму и предложен его естественный аналог для пяти точек.
\end{abstract}

\section*{Форма набора точек}

Давайте построим квадратичную форму $W_{\bm{x}}$ на $\RR^{n-1}$ 
для данного набора из $n$ точек $\bm{x}\z=(x_1,\dots,x_n)$ в метрическом пространстве $X$.

Для этого, выберем симплекс $\triangle$ в $\RR^{n-1}$.
Обозначим через $v_1,\dots,v_n$ его вершины.
Если $(e_1,\dots,e_{n-1})$ канонический базис в $\RR^{n-1}$,
то можно взять $v_i\z=e_i$ при $i<n$ и $v_n=0$.

Обозначим через $|a-b|_X$ расстояние между точками $a$ и $b$ в метрическом пространства $X$.
Заметим, что это равенство
\[W_{\bm{x}}(v_i-v_j)=|x_i-x_j|^2_X\] 
для всех $i$ и $j$ определяет единственную квадратичную форму $W_{\bm{x}}$.

Построенная квадратичная форма $W_{\bm{x}}$ 
будет называться \emph{формой набора $\bm{x}$ по отношению к симплексу $\triangle$}.

Заметим, что набор $\bm{x}=(x_1,\dots,x_n)$ в $X$ изометричен набору евклидова пространства тогда и только тогда, когда 
$W_{\bm{x}}(v)\ge 0$
для любого $v\in \RR^{n-1}$.

В частности, условие $W_{\bm{x}}\ge 0$  для тройки точек означает, что для них
выполняются все три неравенства треугольника. 

\section*{Четырёх-точеное условие Александрова}

Рассмотрим четвёрку точек $\bm{x}\z=(x_1,x_2,x_3,x_4)$ и её форму $W_{\bm{x}}$ относительно тетраэдра $\triangle$ в $\RR^3$.


Из случая $n=3$ 
следует, что форма $W_{\bm{x}}$ 
неотрицательна на каждой плоскости параллельной грани тетраэдра $\triangle$.
В частности, $W_{\bm{x}}$ не может иметь больше одного отрицательного собственного значения.

Предположим $W_{\bm{x}}(w)<0$ для некоторого $w\in\RR^3$.
Из вышесказанного следует, 
что вектор $w$ трансверсален к каждой из четырёх плоскостей параллельных граням $\triangle$.

Рассмотрим проекцию $\triangle$ вдоль $w$ на трансверсальную к нему плоскость. 
Заметим, что в проекции четыре вершины тетраэдра $\triangle$ лежат в общем положении; 
то есть, никакие три из них не лежат на прямой.
Таким  мы видим одну из двух картинок показанных на рисунке.
Легко видеть, что комбинаторика картинки не зависит от выбора $w$.

\begin{wrapfigure}{r}{52mm}
\begin{lpic}[t(-5mm),b(3mm),r(0mm),l(0mm)]{pics/quad(1)}
\lbl[]{11,-2;$\quadra(4)$}
\lbl[]{40,-2;$\quadra(3)$}
\end{lpic}
\end{wrapfigure}

Если мы видим картинку слева, то $\bm{x}$ считается типа 
$\quadra(4)$, а если справа то типа $\quadra(3)$.

Следующее утверждение даёт связь между формами четвёрок и ограничением на кривизну в смысле Александрова.
Доказательства предоставляются читателю.

Предположим $X$ есть полное пространство с внутренней метрикой.
Тогда
\begin{itemize}
\item Если $W_{\bm{x}}\ge 0$ 
для любой четвёрки $\bm{x}=(x_1,\dots,x_4)$ 
тогда $X$ изометричен выпуклому телу в гильбертовом пространстве. 
\item В $X$ нет четвёрок точек типа $\quadra(3)$ тогда и только тогда, когда 
$X$ является александровским пространством неотриательной кривизны, 
далее мы пишем «$\Alex[0]$ пространство».
\item В $X$ нет четвёрок точек типа $\quadra(4)$ тогда и только тогда, когда 
$X$ является $\CAT[0]$ пространством, то есть пространством Адамара.
\end{itemize}

\section*{Пяти-точечные условия}

Попробуем повторить то же для пятёрок точек
 $\bm{x}=(x^1,\dots,x^5)$ в метрическом пространстве.
В этом случае, форма пятёрки $W_{\bm{x}}$ определена на $\RR^4$
и она неотрицательна на каждой плоскости параллельной любой из 10 двумерных граней $4$-мерного симплекса $\triangle$.
В частности $W_{\bm{x}}$ имеет не более двух отрицательных собственных значений.

В случае если $W_{\bm{x}}$ имеет ровно два отрицательных собственных значений,
можно выбрать плоскость $\Pi$ 
такую что сужение $W_{\bm{x}}$ на $\Pi$ отрицательно.
Спроектируем $\triangle$ вдоль $\Pi$ на трансверсальную плоскость.
То же рассуждение, что и для четвёрок даёт, что вершины $\triangle$ лежат в общем положении. 
Значит в проекции мы можем увидеть одну из следующих трёх картинок.

%\begin{center}
\begin{wrapfigure}{r}{70mm}
\begin{lpic}[t(-0mm),b(3mm),r(0mm),l(0mm)]{pics/penta(1)}
\lbl[]{10,-3;$\penta(5)$}
\lbl[]{34,-3;$\penta(4)$}
\lbl[]{57.5,-3;$\penta(3)$}
\end{lpic}
\end{wrapfigure}
%\end{center}
Иначе говоря, для любой пятёрки точек $\bm{x}$,
либо $W_{\bm{x}}$ имеет не более одного отрицательного собственного значения либо в точности два отрицательных собственных значений 
и в последнем случае она принадлежит к одному из трёх типов $\penta(5)$, $\penta(4)$ или $\penta(3)$.

Можно рассматривать метрические пространства в которых нет пятёрок каких-то из этих типов.
Например, назовём $\penta(\hat 3,\hat 4)$ пространством пространство с внутренней метрикой без пятёрок типа $\penta(3)$ и $\penta(4)$.

Вот несколько наблюдений о таких пространствах.

\begin{enumerate}[(i)]
\item\label{i} Любое $\CAT[0]$ пространство является $\penta(\hat 3,\hat 4,\hat 5)$.
Другими словами, форма  для любой пятёрки точек 
в $\CAT[0]$ пространстве имеет не более одного отрицательного собственного значения.
\item\label{ii} $\Alex[0]$ пространства не содержат пятёрок точек типа $\penta(3)$ и $\penta(4)$; 
иначе говоря, они являются $\penta(\hat 3,\hat 4)$ пространствами.
\item\label{iii} Если полное риманово многообразие $M$ не имеет пятёрок точек типа $\penta(5)$ тогда оно является $\CAT[0]$ пространством.
\end{enumerate}
 
\begin{thm}{Вопрос}
Есть ли в $\penta(\hat 3,\hat 4)$ пространствах осмысленная геометрия?
\end{thm}

Этот вопрос мне кажется интерсным поскольку из наблюдений (\ref{i}) и (\ref{ii}) 
следует, что $\penta(\hat 3,\hat 4)$ пространства
включают все $\CAT[0]$ и $\Alex[0]$ пространства, то есть все александровские пространства с неотрицательной и неположительной кривизной.
Значит, положительный ответ 
смог бы дать единый подход к изучению этих пространств.
Из общих свойств $\CAT[0]$ и $\Alex[0]$ пространств можно назвать такие.
\begin{itemize}
\item Если две кратчайшие имеют общие концы и ещё одну общую точку то они совпадают.
\item Плоскость с метрикой индуцированной нормой является $\CAT[0]$ или $\Alex[0]$ только если метрика евклидова.
\end{itemize}
Мне пока не понятно выполняются ли эти свойства в $\penta(\hat 3,\hat 4)$ пространствах.

Вот ещё один вопрос связанный с наблюдениями (\ref{i}) и (\ref{iii}).

\begin{thm}{Вопрос}
Верно ли, что любое полное $\penta(\hat 3,\hat 4,\hat 5)$ пространство с внутренней метрикой является $\CAT[0]$ пространством?
\end{thm} 

\end{document}