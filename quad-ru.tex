\documentclass{article}
\usepackage{quad}
\usepackage[english,russian]{babel}
\usepackage[utf8]{inputenc}

\begin{document}

\title{В поисках пятиточечного условия\\ Александровского типа}
\author{Антон Петрунин}
\date{}
\maketitle
\begin{abstract}
Приведено описание четырёх-точеного условия Александрова через квадратичную форму и предложен его естественный аналог для пяти точек.
\end{abstract}

\section*{Прицепленная форма}

Для набора из $n$ точек $\bm{x}\z=(x_1,\dots,x_n)$ в метрическом пространстве $X$, мы построим квадратичную форму $W_{\bm{x}}$ на $\RR^{n-1}$.

Выберем симплекс $\triangle$ в $\RR^{n-1}$.
Обозначим через $v_1,\dots,v_n$ его вершины.
Если $(e_1,\dots,e_n)$ канонический базис в $\RR^{n-1}$,
то можно предположить $v_i\z=e_i$ при $i<n$ и $v_n=0$.

Обозначим через $|a-b|_X$ расстояние между точками $a$ и $b$ в метрическом пространства $X$.
Заметим, что это равенство
\[W_{\bm{x}}(v_i-v_j)=|x_i-x_j|^2_X\] 
для всех $i$ и $j$ полностью определяет квадратичную форму $W_{\bm{x}}$.

Построенная квадратичная форма $W_{\bm{x}}$ будет называться \emph{прицепленной} к набору $\bm{x}$ по отношению к симплексу $\triangle$.

Заметим, что набор $\bm{x}=(x_1,\dots,x_n)$ в $X$ изометричен набору евклидова пространства тогда и только тогда, когда 
$W_{\bm{x}}(v)\ge 0$
для любого $v\in \RR^{n-1}$.

В частности, при $n=3$ условие $W_{\bm{x}}\ge 0$ означает, что для точек $x_1$, $x_2$ и $x_3$,
выполняются все три неравенства треугольника. 

\section*{Четырёх-точеное условие Александрова}

Рассмотрим четвёрку точек $\bm{x}\z=(x_1,\dots,x_4)$ и прицепленную к ней форму $W_{\bm{x}}$ относительно тетраэдра $\triangle$ в $\RR^3$.


Из случая $n=3$ 
следует, что форма $W_{\bm{x}}$ 
неотрицательна на каждой плоскости параллельной грани тетраэдра $\triangle$.
В частности, $W_{\bm{x}}$ не может иметь больше одного отрицательного собственного значения.

Предположим $W_{\bm{x}}(w)<0$ для некоторого $w\in\RR^3$.
Из вышесказанного следует, что 
$w$ трансверсален к каждой из четырёх плоскостей параллельных граням $\triangle$.
Спроектируем $\triangle$ вдоль $w$ на трансверсальную к нему плоскость. 
Заметим, что в проекции четыре вершины тетраэдра $\triangle$ лежат в общем положении; 
то есть, никакие три из них не лежат на прямой.
Таким образом в проекции мы можем увидеть одну из двух картинок показанных на рисунке.
Легко видеть, что комбинаторика картинки не зависит от выбора $w$.

\begin{wrapfigure}{r}{52mm}
\begin{lpic}[t(-5mm),b(3mm),r(0mm),l(0mm)]{pics/quad(1)}
\lbl[]{11,-2;$\quadra(4)$}
\lbl[]{40,-2;$\quadra(3)$}
\end{lpic}
\end{wrapfigure}

Если мы видим картинку слева, то $\bm{x}$ считается типа 
$\quadra(4)$, а если справа то типа $\quadra(3)$.

Следующее утверждение даёт связь между прицепленными формами и ограничением на кривизну в смысле Александрова.
Доказательства предоставляются читателю.

Предположим $X$ есть полное пространство с внутренней метрикой.
Тогда
\begin{itemize}
\item Если $W_{\bm{x}}\ge 0$ 
для любой четвёрки $\bm{x}=(x_1,\dots,x_4)$ 
тогда $X$ изометричен выпуклому телу в гильбертовом пространстве. 
\item В $X$ нет четвёрок точек типа $\quadra(3)$ тогда и только тогда, когда 
$X$ имеет неотрицательную кривизну в смысле Александрова.
\item В $X$ нет четвёрок точек типа $\quadra(4)$ тогда и только тогда, когда 
$X$ является $\CAT[0]$ пространством.
\end{itemize}

\section*{Пяти-точечные условия}

Попробуем повторить то же для пяти-точечных наборов
 $\bm{x}=(x^1,\dots,x^5)$ в метрическом пространстве.
В этом случае, прицепленная форма $W_{\bm{x}}$ определена на $\RR^4$
и она неотрицательна на каждой плоскости параллельной любой из 10 двумерных граней $4$-мерного симплекса $\triangle\subset \RR^4$.
В частности $W_{\bm{x}}$ имеет не более двух отрицательных собственных значений.

В случае если $W_{\bm{x}}$ имеет ровно два отрицательных собственных значений,
можно выбрать плоскость $\Pi$ 
такую что сужение $W_{\bm{x}}$ на $\Pi$ отрицательно.
Спроектируем $\triangle$ вдоль $\Pi$ на трансверсальную плоскость.
То же рассуждение, что и в случает $n=4$ даёт, что вершины $\triangle$ лежат в общем положении. 
Значит в проекции мы можем увидеть одну из следующих трёх картинок.

%\begin{center}
\begin{wrapfigure}{r}{70mm}
\begin{lpic}[t(-0mm),b(3mm),r(0mm),l(0mm)]{pics/penta(1)}
\lbl[]{10,-3;$\penta(5)$}
\lbl[]{34,-3;$\penta(4)$}
\lbl[]{57.5,-3;$\penta(3)$}
\end{lpic}
\end{wrapfigure}
%\end{center}
Иначе говоря, для любой пятёрки точек $\bm{x}\z=(x_1,\dots,x_5)$,
либо $W_{\bm{x}}$ имеет не более одного отрицательного собственного значения либо в точности два отрицательных собственных значений 
и в последнем случае она принадлежит к одному из трёх типов $\penta(5)$, $\penta(4)$ или $\penta(3)$.

Можно рассматривать метрические пространства в которых нет пятёрок каких-то из этих типов.
Например, обозначим через $\penta(\hat 3,\hat 4)$ класс пространств с внутренней метрикой без пятёрок типа $\penta(3)$ и $\penta(4)$.

Вот несколько наблюдений о таких пространствах.

\begin{enumerate}[(i)]
\item\label{i} Любое $\CAT[0]$ пространство является $\penta(\hat 3,\hat 4,\hat 5)$.
Другими словами,
прицепленная форма $W_{\bm{x}}$ для любой пятёрки точек $\bm{x}$ 
в $\CAT[0]$ пространстве имеет не более одного отрицательного собственного значения.
\item\label{ii} Пространства с неотрицательной кривизной в смысле Александрова не содержат пятёрок точек типа $\penta(3)$ и $\penta(4)$; 
иначе говоря, они являются $\penta(\hat 3,\hat 4)$ пространствами.
\item\label{iii} Если полное риманово многообразие $M$ не имеет пятёрок точек типа $\penta(5)$ тогда оно является $\CAT[0]$ пространством.
\end{enumerate}

Из наблюдений (\ref{i}) и (\ref{ii}) 
следует, что $\penta(\hat 3,\hat 4)$ пространства
включают все $\CAT[0]$ пространства 
и пространства с неотрицательной кривизной в смысле Александрова.

\begin{thm}{Вопрос}
Есть ли в $\penta(\hat 3,\hat 4)$ пространствах осмысленная геометрия?
\end{thm}

В частности, мы не знаем ответа на следующий вопрос.

\begin{thm}{Вопрос}
Пусть $X$ полное $\penta(\hat 3,\hat 4)$ пространство с внутренней метрикой.
Предположим две кратчайшие в $X$ проходят через три различные точки.
Верно ли, что эти точки лежат в дуге общей для обоих кратчайших?
\end{thm}

\begin{thm}{Вопрос}
Пусть $X$ координатная плоскость с метрикой индуцированной нормой.
Предположим $X$ образует $\penta(\hat 3,\hat 4)$ пространство.
Верно ли, что $X$ изометрична евклидовой плоскости?
\end{thm}

Вот ещё один вопрос связанный с наблюдениями (\ref{i}) и (\ref{iii}).

\begin{thm}{Вопрос}
Верно ли, что любое полное $\penta(\hat 3,\hat 4,\hat 5)$ пространство с внутренней метрикой является $\CAT[0]$ пространством?
\end{thm} 

\end{document}