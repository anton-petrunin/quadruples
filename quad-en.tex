\documentclass{article}
\usepackage{quad}

\begin{document}

\title{A quest for 5-point condition\\ of Alexandrov's type}
\author{Anton Petrunin}
%\address{Anton Petrunin, Math. Dept., PSU,University Park, PA 16802, USA.}\address{petrunin@math.psu.edu}
\date{}
\maketitle
\begin{abstract}
I give a description of Alexandrov 4-point comparison
via quadratic forms
and then propose a natural 5-point condition which might have future.
\end{abstract}


\section*{Associated form}

Let us construct a quadratic form $W_{\bm{x}}$ on $\RR^{n-1}$ 
for given $n$-point array $\bm{x}\z=(x_1,\dots,x_n)$ in a metric space $X$.

Fix a nondegenerate simplex $\triangle$ in $\RR^{n-1}$.
Denote by $v_1,\dots,v_n$ its vertices.
If $(e_1,\dots,e_{n-1})$ is the standard basis on $\RR^{n-1}$,
we may assume that $v_i=e_i$ for $i<n$ and $v_n=0$.

Let us denote by $|a-b|_X$ the distance between points $a$ and $b$ in the metric space $X$.
Set
\[W_{\bm{x}}(v_i-v_j)=|x_i-x_j|^2_X\] 
for all $i$ and $j$.
Note that this identity define $W_{\bm{x}}$ uniquely.


The constructed quadratic form $W_{\bm{x}}$ will be called \emph{form of the point array $\bm{x}$ with respect to the simplex $\triangle$}.

Note that an array $\bm{x}=(x_1,\dots,x_n)$ in a metric space $X$ is isometric to an array in Euclidean space if and only if 
$W_{\bm{x}}(v)\ge 0$
for any $v\in \RR^{n-1}$.

In particular,  the
condition $W_{\bm{x}}\ge 0$ for triples of points means that 
all three triangle inequalities hold. 

\section*{Alexandrov's 4-point comparison}

Now let us discuss the relation between form of quadruples
and geometry of the space.
In this case $\triangle$ is a tetrahedron on $\RR^3$.


From the $3$-point case, 
it follows that $W_{\bm{x}}$ 
is nonnegative on every plane parallel to a face of the tetrahedron $\triangle$.
In particular, $W_{\bm{x}}$ can have at most one negative eigenvalue.

Assume $W_{\bm{x}}(w)<0$ for some $w\in\RR^3$.
It follows from above that
$w$ is transversal to each of 4 planes parallel to a faces of $\triangle$.

Consider the projection of $\triangle$ along $w$ to a transversal plane. 
Note that in the projection the 4 vertices of $\triangle$ lie in general position; 
that is, no three of them lie on one line.
Therefore  we can see one of two combinatorial pictures shown on the diagram.
It is easy to see that the combinatorics of the picture does not depend on the choice of $w$.

\begin{wrapfigure}{r}{52mm}
\begin{lpic}[t(-5mm),b(3mm),r(0mm),l(0mm)]{pics/quad(1)}
\lbl[]{11,-2;$\quadra(4)$}
\lbl[]{40,-2;$\quadra(3)$}
\end{lpic}
\end{wrapfigure}

If we see the diagram on the left we say that $\bm{x}$ is 
of type $\quadra(4)$ and otherwise we say that it is of type $\quadra(3)$.
 

The following statements give a connection between the forms $W_{\bm{x}}$ of the quadruple $\bm{x}$
and the curvature bounds in the sense of Alexandrov.
Their proofs are left to the reader.

Assume $X$ is a complete space with intrinsic metric.
Then
\begin{itemize}
\item If $W_{\bm{x}}\ge 0$ 
for any quadrilateral $\bm{x}=(x_1,\dots,x_4)$ 
then $X$ is isometric to a closed convex set in a Hilbert space. 
\item $X$ has no quadruples of type $\quadra(3)$ if and only if 
$X$ has nonnegative curvature in the sense of Alexandrov, further we say ``$\Alex[0]$ space''.
\item $X$ has no quadruples of type $\quadra(4)$ if and only if 
$X$ is a $\CAT[0]$ space which is also called Hadamard space
\end{itemize}

\section*{5-point conditions}

Let us try to do the same 
for 5-points arrays $\bm{x}=(x^1,\dots,x^5)$ in a metric space.
Its form $W_{\bm{x}}$ is defined on $\RR^4$
and it has to be nonnegative on any plane which is parallel to any of 10 two-dimensional faces of the $4$-simplex $\triangle$.
In particular $W_{\bm{x}}$ has at most two negative eigenvalues.

In the case if $W_{\bm{x}}$ has exactly two negative eigenvalues,
one can choose a plane $\Pi$ 
such that the restriction of $W_{\bm{x}}$ to $\Pi$ is negative.
Let us project $\triangle$ along $\Pi$ to a transversal plane.
The same argument as in case $n=4$ shows that after projection
the vertexes of $\triangle$  lie in general position.
Therefore we may get one of the following three combinatorial pictures.

%\begin{center}
\begin{wrapfigure}{r}{70mm}
\begin{lpic}[t(-3mm),b(3mm),r(0mm),l(0mm)]{pics/penta(1)}
\lbl[]{10,-3;$\penta(5)$}
\lbl[]{34,-3;$\penta(4)$}
\lbl[]{57.5,-3;$\penta(3)$}
\end{lpic}
\end{wrapfigure}
%\end{center}
That is, for any 5-point array $\bm{x}$
either $W_{\bm{x}}$ has at most one negative eigenvalue
or it has exactly two negative eigenvalues and belongs to one of these three types $\penta(5)$, $\penta(4)$ or $\penta(3)$.

We may consider the metric spaces
which do not have a 5-point arrays of some of these types.
For example, a $\penta(\hat 3,\hat 4)$ space is a complete length-metric spaces 
without 5-point arrays of type $\penta(3)$ and $\penta(4)$.

Let us list some easy observations about these new classes of metric spaces.

\begin{enumerate}[(i)]
\item\label{i} Any $\CAT[0]$ space is a $\penta(\hat 3,\hat 4,\hat 5)$ space.
In other words,
the form $W_{\bm{x}}$ for any 5-point array $\bm{x}$ in any $\CAT[0]$ space 
has at most one negative eigenvalue.
\item\label{ii} Any $\Alex[0]$ space  has no  5-point arrays of type $\penta(3)$ and $\penta(4)$; 
that is, it is a $\penta(\hat 3,\hat 4)$ space.
\item\label{iii} If a complete Riemannian manifold $M$ has no
5-point array of type $\penta(5)$ then it is a simply connected and it has   non-positive sectional curvature.
\end{enumerate}

\begin{thm}{Question}
 Do $\penta(\hat 3,\hat 4)$ spaces have meaningful geometry?
\end{thm}

I find this question interesting 
because $\penta(\hat 3,\hat 4)$ spaces include  all $\CAT[0]$ as well as $\Alex[0]$ spaces; that is Alexandrov spaces with nonnegative and nonpositive curvature.
The latter statement follows from the observations (\ref{i}) and (\ref{ii}).
Therefore a positive answer to the above question might lead to a uniform treatment of these two types of spaces.

Here are couple of examples of properties shared by $\CAT[0]$ and $\Alex[0]$ spaces:
\begin{itemize}
\item Two minimizing geodesics with common ends and yet one common point have to coinside.
\item A plane with metric induced by norm is $\CAT[0]$ or $\Alex[0]$ only if the norm is Euclidean.
\end{itemize}
At the moment I do now know if the same holds for $\penta(\hat 3,\hat 4)$ spaces.

Here is an other question related to the observations (\ref{i}) and (\ref{iii}).

\begin{thm}{Question}
Is it true that any $\penta(\hat 3,\hat 4,\hat 5)$ space with intrinsic metric has to be $\CAT[0]$?
\end{thm} 

\end{document}